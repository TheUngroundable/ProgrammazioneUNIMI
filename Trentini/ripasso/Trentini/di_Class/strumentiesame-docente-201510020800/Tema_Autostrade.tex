\documentclass[a4paper]{article}

\usepackage{verbatim}
%\usepackage[italian]{babel}
\usepackage[a4paper,left=1cm,right=1cm,top=1cm]{geometry}
%\usepackage[a4paper,left=1.9cm,right=1.9cm,top=1.5cm]{geometry}
%\usepackage[compact]{titlesec}
% \titlespacing{\section}{0pt}{*1}{*1}
% \titlespacing{\subsection}{10pt}{*1}{*1}
% \titlespacing{\subsubsection}{10pt}{*1}{*1}
%\usepackage{mdwlist}
%\setlength{\columnsep}{1.38cm}
\setlength\parindent{0pt}
%\setlength\bibsep{0pt}
\setlength{\parskip}{0pt}
\setlength{\parsep}{0pt}
%\usepackage[italian]{babel}
%\usepackage[latin1]{inputenc}
\usepackage[utf8]{inputenc}
%\usepackage{paralist}
\usepackage{graphicx}
\usepackage{paralist}

% \topmargin=-2.7cm
% \oddsidemargin=-1cm
% \evensidemargin=-1cm
% \textwidth=18cm
\textheight=26.7cm

%%%%%%%%%%%%%%%%%%%%%%%%%%%%%%%%%%%%%%%%%%%%%%%%%%%%%%%%%%%%%%%%%%%%%%%
\title{Laboratorio di Programmazione\\
Edizione 1 - Turni A, B, C}

\date{\textit{ESAME del 21 Settembre 2015}}
%\textbf{BOZZA}}
%\makeindex

%%%%%%%%%%%%%%%%%%
\begin{document}
%\pagestyle{empty}
%\twocolumn[\begin{center}\textbf{LabProg 2012-2013 - TESTO ESAME\\
%Andrea Trentini - D.I. - UniMi\\
%05 febbraio 2013} \end{center} \vspace{.5cm}]
\maketitle

%\tableofcontents
%\hrule

\hrulefill

\textbf{Avvertenze}

\begin{compactitem}
\item 
Nello svolgimento dell'elaborato è possibile
usare qualunque classe delle librerie standard di Java.

\item Non è invece
ammesso l'uso delle classi del package {\tt prog} allegato al libro di
testo del Prof.~Pighizzini e impiegato nella prima parte del corso.

\item Si consiglia CALDAMENTE l'utilizzo dello script ``checker.sh'' per 
compilare ed effettuare una prima valutazione del proprio elaborato.
Si consiglia anche di leggere il sorgente dei \texttt{Test\_*.java} per 
capire cosa devono offrire le classi da sviluppare.

\item Ricordarsi, quando si programma: \emph{Repetita NON iuvant} o DRY (\emph{Don't Repeat Yourself}).

\end{compactitem}


%%%%%%%%%%%%%%%%%%
\hrulefill
\section*{Tema d'esame}

Lo scopo dell'esercizio è realizzare un modello semplificato per la gestione di 
strade a pedaggio e non. Quindi avremo una gerarchia di strade (Strada, 
StradaPedaggio, etc.) e una gerarchia di veicoli (Veicolo, Auto, Moto, Tir, 
Pullmann, etc.) e le classi dovranno esporre metodi per il calcolo del 
pedaggio, della velocità media, etc.

Le \textbf{classi} da realizzare sono le seguenti (dettagli nelle sezioni 
successive):
\medskip

\begin{compactenum}
\item \texttt{Strada}: strada generica, non prevede pedaggio
\item \texttt{StradaPedaggio}: generica strada a pedaggio
\item \texttt{Autostrada}: sottoclasse di StradaPedaggio, introduce il concetto 
di ``veicolo ammesso al transito'' (deve avere una certa potenza)
\item \texttt{Ticket}: biglietto di ingresso, tiene traccia dell'orario di 
ingresso

\item \texttt{Veicolo}: veicolo generico, classe astratta
\item \texttt{Auto}: sottoclasse di Veicolo
\item \texttt{Tir}: sottoclasse di Veicolo

\item \texttt{MultaException}: eccezione che rappresenta l'evento di 
superamento del limite di velocità


\end{compactenum}

%%%%%%%%%%%%%%%%%%
\section*{Specifica delle classi}

Le classi (\textbf{pubbliche}!) dovranno esporre almeno i metodi e costruttori \textbf{pubblici} specificati, più eventuali altri metodi e costruttori %\textit{privati}, 
se ritenuti opportuni.
%in alcuni casi le definizioni dei metodi sono incomplete
%(vanno aggiunti i tipi mancanti).
Gli attributi (campi) delle classi devono essere \textbf{privati}.
per leggere e modificarne i valori, 
creare opportunamente, e solo dove necessario, i metodi di accesso ({\tt set} e 
{\tt get}).
Se si usano classi che utilizzano tipi generici, si suggerisce  di utilizzarne 
le versioni opportunamente istanziate (es. \texttt{ArrayList<String>} invece di
\texttt{ArrayList}).
Ogni classe deve avere il metodo {\tt 
toString} che rappresenti lo stato delle istanze. 

% Alcuni controlli di coerenza vengono suggeriti nel testo, potrebbero
% essercene altri a discrezione. 
% Si consiglia di posporre
% l'implementazione dei controlli di coerenza, come ultima operazione,
% dopo aver realizzato un sistema funzionante.

%%%%%%%%%%%%%%%%%%
\subsection*{public class Strada}

Deve definire gli attributi: \texttt{int limite (km/h); int lunghezza 
(km)}.

Definire almeno un costruttore che permetta di impostare gli attributi.

E i seguenti metodi (oltre ai get per gli attributi, NON implementare i set, 
basta il costruttore):

\begin{compactitem}

\item\texttt{public float orePercorrenzaVelocitaCodice()} restituisce quante 
ore ci vogliono a percorrere tutta la lunghezza della strada andando alla 
velocità massima possibile (limite di velocità). Nota: il float restituito rappresenta
ore e frazioni di ore, ad esempio 1.20 rappresenta un'ora e 12 minuti.

\item\texttt{public float velocitaMediaDatoTempoPercorrenzaInSec(float 
percorrenza)} restituisce la velocità media tenuta dato un tempo di percorrenza 
effettivo espresso in secondi.

\item\texttt{public boolean superatoLimiteDatoTempoPercorrenzaInSec(float 
percorrenza)} restituisce \textit{true} se dato il tempo di percorrenza (in secondi) si può 
supporre che il limite di velocità sia stato superato.

\end{compactitem}





%%%%%%%%%%%%%%%%%%
\subsection*{public class StradaPedaggio}

Sottoclasse di Strada, aggiunge la gestione del pedaggio.

Deve definire gli attributi: \texttt{float tariffaBase (euro/km).}

Almeno un costruttore che permetta di impostare gli attributi.

E i seguenti metodi (oltre ai get per gli attributi, NON implementare i set, 
basta il costruttore):

\begin{compactitem}
\item \texttt{public float pedaggio(Veicolo v)} calcola il pedaggio (in euro) in funzione 
della tariffa base, della lunghezza della strada e del numero di assi del veicolo: fino a 3 assi, tariffa 
base; oltre i 3 assi, 1.5*(tariffa base).
\end{compactitem} 




%%%%%%%%%%%%%%%%%%
\subsection*{public class Autostrada}

Sottoclasse di StradaPedaggio, aggiunge la gestione dell'accesso (accedono 
solo i veicoli sopra una certa potenza) e la gestione degli ingressi (tiene 
traccia di chi \`e dentro, calcola se sono stati superati i limiti, etc.).

Deve definire gli attributi: \texttt{float 
potenzaMinimaPerAccedere (kW)} e un container per tenere traccia dei veicoli 
entrati.

Almeno un costruttore che permetta di impostare gli attributi.

E i seguenti metodi (oltre ai get per gli attributi, NON implementare i set, 
basta il costruttore):

\begin{compactitem}
\item \texttt{public float pedaggio(Veicolo v)} override di pedaggio della 
superclasse, accetta solo veicoli con potenza superiore a potenzaMinima... e 
calcola il pedaggio come sopra; se non accetta, lancia un'eccezione.

\item \texttt{public Ticket ingresso(Veicolo v)} se il veicolo \`e ammesso
(controllo su potenza), lo lascia entrare ed emette un ticket (vedi classe relativa) per il 
veicolo stesso; lancia un'eccezione se il veicolo non \`e ammesso.
Nota: java.lang.System.currentTimeMillis() permette di avere l'ora esatta espressa 
in millisecondi.

\item \texttt{public float uscita(Veicolo v)} se il veicolo non \`e presente tra 
i veicoli entrati, lancia un'eccezione; se il tempo di percorrenza (calcolato in base
all'ora attuale e all'ora di ingresso) \`e inferiore a 
quello ``legale'' (rispettando il limite di velocit\`a), lancia MultaException (vedi classe 
relativa), altrimenti fa uscire il veicolo, calcola il pedaggio e lo restituisce.


\item \texttt{public int quantiVeicoli()} restituisce il numero di veicoli 
attualmente in viaggio.

\item \texttt{public float potenzaMedia()} calcola e restituisce la potenza 
media dei veicoli attualmente in viaggio.


\end{compactitem} 



%%%%%%%%%%%%%%%%%%
\subsection*{public abstract class Veicolo}

Deve definire gli attributi:     \texttt{String targa;  int assi (numero di); 
float potenza (in kW)}. Inoltre deve avere un reference a Ticket (vedi classe 
relativa) con i metodi set e get relativi.


%%%%%%%%%%%%%%%%%%
\subsection*{public class Auto}

Sottoclasse di Veicolo, non ha attributi aggiuntivi, implementa un costruttore 
che fissa il numero di assi a 2.


%%%%%%%%%%%%%%%%%%
\subsection*{public class Tir}


Sottoclasse di Veicolo, non ha attributi aggiuntivi, implementa un costruttore 
che fissa il numero di assi a 5.


%%%%%%%%%%%%%%%%%%
\subsection*{public class Ticket}

Deve definire un attributo  \texttt{long timestamp} (per registrare l'orario di ingresso) e 
avere un reference a Veicolo, coi relativi set e get.

\textbf{Suggerimento:} per il timestamp usare \texttt{java.lang.System.currentTimeMillis()},
che restituisce l'ora corrente espressa in millisecondi.

%%%%%%%%%%%%%%%%%%
\subsection*{public class Main}


Creare un main che realizzi i seguenti:

\begin{compactenum}
\item istanzi una Autostrada (attributi: 50 km lunghezza, 100 km/h 
limite vel., 0.20 euro/km pedaggio, 50 kW potenza minima per accedere)
\item istanzi una serie di veicoli prendendo i dati da un file di testo (vedi 
sotto)
\item li immetta nell'autostrada
\item stampi la situazione dell'autostrada dopo ogni ingresso
\item calcoli la potenza media dei veicoli che sono effettivamente entrati
\item li faccia uscire uno a uno dall'autostrada, stampando la situazione
dopo ogni uscita
\end{compactenum}

\textbf{Formato del file di testo:}  TipoVeicolo;Targa;Potenza

Utilizzare il seguente contenuto:

\begin{quote}
Tir;MI2121212;600\\
Auto;PV77777;60\\
Tir;CO98989;900\\
Auto;MI656565;200\\
Auto;MI43432;40\\
Tir;MI2323232;600\\
Tir;MI373738;800
\end{quote}

(potete usare il file \texttt{autostrada.txt} allegato al tema)

%%%%%%%%%%%%%%%%%%%%%%%%%%%%%%%%%%%%%%%%%%%%%%%%%%%%%%%
\hrulefill
\section*{Raccomandazioni}

Affinché l'elaborato sia valutato, è richiesto che sia le classi sviluppate che i test risultino
\textit{compilabili}. A tal fine i metodi/costruttori che non saranno sviluppati dovranno comuque avere
una implementazione fittizia come la seguente:
\begin{verbatim}
public void nomeDelMetodo () {
   throw new UnsupportedOperationException();
}
\end{verbatim}
Si suggerisce quindi di dotare da subito le classi di tutti i metodi richiesti, 
implementandoli in modo fittizio, e poi di sostituire man mano le 
implementazioni fittizie con implementazioni che rispettino le specifiche.

\hrulefill

%%%%%%%%%%%%%%%%%%
\section*{Consegna}
%\marginpar{QUESTA SEZIONE E' PRESENTE PER COMPLETEZZA}
Si ricorda che le classi devono essere tutte \textit{public} e che vanno 
consegnati tutti (e soli) i file \textit{.java} prodotti.
NON vanno consegnati i \textit{.class}.
NON vanno consegnati i file relativi al meccanismo di autovalutazione 
(\textit{Test\_*.java, AbstractTest.java, *.sh}).
Per la consegna, eseguite l'upload dei SINGOLI file sorgente (NON un file archivio!) dalla pagina web: http://upload.di.unimi.it
nella sessione del vostro docente.

\hrulefill

\begin{center}*** ATTENZIONE!!! ***\end{center}

NON VERRANNO VALUTATI GLI ELABORATI CON ERRORI DI COMPILAZIONE O 
LE CONSEGNE CHE NON RISPETTANO LE SPECIFICHE (ad esempio consegnare un 
archivio zippato è sbagliato, come anche consegnare ad un docente diverso dal 
proprio assegnato).

UN SINGOLO ERRORE DI COMPILAZIONE O DI PROCEDURA INVALIDA 
\textbf{TUTTO} 
L'ELABORATO.
\medskip

\hrulefill

{\bf Per ritirarsi}
fare l'upload di un file vuoto di nome  \texttt{ritirato.txt}. Se avete caricato 
dei file nella sessione del docente sbagliato, caricate lì un file vuoto di 
nome  \texttt{errataConsegna.txt} e caricate poi i file nella sessione giusta. 

\hrulefill

%%%%%%%%%%%%%%
\end{document}
