\documentclass[a4paper]{article}

\usepackage{verbatim}
%\usepackage[italian]{babel}
\usepackage[a4paper,left=2cm,right=2cm,top=2cm]{geometry}
%\usepackage[a4paper,left=1.9cm,right=1.9cm,top=1.5cm]{geometry}
%\usepackage[compact]{titlesec}
% \titlespacing{\section}{0pt}{*1}{*1}
% \titlespacing{\subsection}{10pt}{*1}{*1}
% \titlespacing{\subsubsection}{10pt}{*1}{*1}
%\usepackage{mdwlist}
%\setlength{\columnsep}{1.38cm}
\setlength\parindent{0pt}
%\setlength\bibsep{0pt}
\setlength{\parskip}{0pt}
\setlength{\parsep}{0pt}
%\usepackage[italian]{babel}
%\usepackage[latin1]{inputenc}
\usepackage[utf8]{inputenc}
%\usepackage{paralist}
\usepackage{graphicx}
\usepackage{paralist}

 \topmargin=-2.7cm
% \oddsidemargin=-1cm
% \evensidemargin=-1cm
% \textwidth=18cm
\textheight=26.7cm

%%%%%%%%%%%%%%%%%%%%%%%%%%%%%%%%%%%%%%%%%%%%%%%%%%%%%%%%%%%%%%%%%%%%%%%
\title{Laboratorio di Programmazione\\
Edizione 1 - Turni A, B, C}

\date{\textit{ESAME del 10 Luglio 2015}}
%\makeindex

%%%%%%%%%%%%%%%%%%
\begin{document}
%\pagestyle{empty}
%\twocolumn[\begin{center}\textbf{LabProg 2012-2013 - TESTO ESAME\\
%Andrea Trentini - D.I. - UniMi\\
%05 febbraio 2013} \end{center} \vspace{.5cm}]
\maketitle

%\tableofcontents
%\hrule
\textbf{Avvertenze}

\begin{compactitem}
\item 
Nello svolgimento dell'elaborato è possibile
usare qualunque classe delle librerie standard di Java.

\item Non è invece
ammesso l'uso delle classi del package {\tt prog} allegato al libro di
testo del Prof.~Pighizzini e impiegato nella prima parte del corso.

\item Si consiglia CALDAMENTE l'utilizzo dello script ``checker.sh'' per 
compilare ed effettuare una prima valutazione del proprio elaborato.
Si consiglia anche di leggere il sorgente dei \texttt{Test\_*.java} per 
capire cosa devono offrire le classi da sviluppare.

\item Ricordarsi, quando si programma: \emph{Repetita NON iuvant} o DRY (\emph{Don't Repeat Yourself}).


\end{compactitem}


%%%%%%%%%%%%%%%%%%
\section*{Tema d'esame}

Lo scopo dell'esercizio è realizzare un'applicazione che gestisca la 
sottomissione di articoli ad una conferenza.

Una conferenza ha un titolo e alcuni vincoli per %il contenuto de
gli articoli 
che vengono proposti, tipicamente un numero di pagine (dati il numero di righe per 
pagina e il numero di caratteri per riga).

Il sistema della conferenza deve accettare gli articoli solo se rispettano i 
vincoli, poi deve poter assegnare un revisore agli articoli che ne sono ancora 
privi.


Le \textbf{classi} da realizzare sono le seguenti (dettagli nelle sezioni 
successive):
\medskip

\begin{compactitem}

\item \texttt{Conferenza}: la classe principale, conterrà gli articoli e 
fornirà alcuni servizi di query

\item \texttt{Articolo}: rappresenta un testo con un autore e un titolo

\item \texttt{Persona}: classe ASTRATTA, rappresenta una persona  generica con 
nome, cognome e mail

\item \texttt{Autore}: sottoclasse di Persona, oltre agli attributi di Persona 
ha anche un ente di appartenenza

\item \texttt{Revisore}: sottoclasse di Persona, oltre agli attributi di 
Persona ha anche un campo ``competenza''

\end{compactitem}
%\end{compactitem}

\section*{Specifica delle classi}

Le classi (\textbf{pubbliche}!) dovranno esporre almeno i metodi e costruttori \textbf{pubblici} specificati, più eventuali altri metodi e costruttori %\textit{privati}, 
se ritenuti opportuni.
%in alcuni casi le definizioni dei metodi sono incomplete
%(vanno aggiunti i tipi mancanti).
Gli attributi (campi) delle classi devono essere \emph{privati}.
per leggere e modificarne i valori, 
creare opportunamente, e solo dove necessario, i metodi di accesso ({\tt set} e 
{\tt get}).
Se si usano tipi generici, si suggerisce  di utilizzarne le versioni
opportunamente istanziate (es. \texttt{ArrayList<String>} invece di
\texttt{ArrayList}).
Ogni classe deve avere il metodo {\tt 
toString} che rappresenti lo stato delle istanze. 

% Alcuni controlli di coerenza vengono suggeriti nel testo, potrebbero
% essercene altri a discrezione. 
% Si consiglia di posporre
% l'implementazione dei controlli di coerenza, come ultima operazione,
% dopo aver realizzato un sistema funzionante.

\subsection*{public class Conferenza}

Ha come attributi: titolo della conferenza, numero di pagine massimo per gli 
articoli, numero di righe per pagina, numero di caratteri per riga.
Inoltre deve poter contenere gli articoli, usare un ``container'' di propria 
scelta.
Oltre agli opportuni costruttori deve disporre dei seguenti metodi pubblici:

\begin{compactitem}


\item\texttt{public Articolo[] articoliDiUnAutore(Autore aut)}
restituisce gli articoli di un singolo autore

\item\texttt{public String[] autori()}
restituisce l'elenco degli autori


\item\texttt{public int 	nrArticoli()}
restituisce il numero di articoli

\item\texttt{public int quantiArticoliNonHannoRevisore()}
restituisce quanti articoli non hanno ancora un revisore assegnato

\item\texttt{public int submit(Articolo art)}
verifica che l'articolo rispetti i vincoli; se s\'i, accetta e restituisce l'ID (si usi come ID la posizione nel `container'), 
altrimenti '-1'


\item\texttt{public String[] titoli()}
restituisce l'elenco dei titoli degli articoli



\end{compactitem}


\subsection*{public class Articolo}

Ha come attributi: titolo dell'articolo, autore (reference all'Autore), testo 
(verrà caricato da un file), revisore (reference al Revisore assegnato).
Deve avere almeno il seguente costruttore:

\begin{compactitem} 


\item \texttt{public Articolo(String filename)}
legge il file ed estrae le info (prima riga del file = titolo,  seconda 
riga = autore, resto del file = testo)


\end{compactitem} 

Inoltre deve disporre dei seguenti metodi pubblici:

\begin{compactitem} 

\item \texttt{public Autore getAutore()} restituisce l'Autore


\item \texttt{public Revisore 	getRevisore()} restituisce il Revisore 
assegnato, null se non è stato assegnato

\item \texttt{public String getTitolo()} restituisce il titolo dell'articolo

\item \texttt{int 	pagine(int righePerPagina, int caratteriPerRiga)} 
restituisce il numero di pagine calcolato in base al testo e al numero di righe per 
pagina e al numero di caratteri per riga

\item \texttt{public int righe(int caratteriPerRiga)} restituisce il numero di 
righe dato un numero di caratteri per riga


\item \texttt{public void 	setRevisore(Revisore rev)} assegna un Revisore


\end{compactitem}








\subsection*{public abstract class Persona}

Classe astratta che rappresenta una persona generica, ha come attributi: nome, 
cognome e mail. I campi non devono essere vuoti (``'') né nulli 
(\texttt{null}), in particolare il campo mail deve contenere almeno il 
carattere ``@'', se non sono soddisfatti questi vincoli va lanciata una 
eccezione.

Definire costruttori e set\&get opportuni.


\subsection*{public class Autore}

Sottoclasse di Persona, aggiunge solo il campo 'ente' che non deve essere né 
vuoto né nullo, lanciare eccezione se non sono verificati questi vincoli.

Definire costruttori e set\&get opportuni.



\subsection*{public class Revisore}

Sottoclasse di Persona, aggiunge solo il campo 'competenze' che non deve essere 
né vuoto né nullo, lanciare eccezione se non sono verificati questi vincoli.

Definire costruttori e set\&get opportuni.



%%%%%%%%%%%%%%%%%%%%%%%%%%%%%%%%%%%%%%%%%%%%%%%%%%%%%%%
\section*{Raccomandazioni}

Affinché l'elaborato sia valutato, è richiesto che sia le classi sviluppate che i test risultino
\textit{compilabili}. A tal fine i metodi/costruttori che non saranno sviluppati dovranno comuque avere
una implementazione fittizia come la seguente:
\begin{verbatim}
public void nomeDelMetodo () {
   throw new UnsupportedOperationException();
}
\end{verbatim}
Si suggerisce quindi di dotare da subito le classi di tutti i metodi richiesti, 
implementandoli in modo fittizio, e poi di sostituire man mano le 
implementazioni fittizie con implementazioni che rispettino le specifiche.
 
\section*{Consegna}

%\marginpar{QUESTA SEZIONE E' PRESENTE PER COMPLETEZZA}

Si ricorda che le classi devono essere tutte \textit{public} e che vanno 
consegnati tutti (e soli) i file \textit{.java} prodotti.
NON vanno consegnati i \textit{.class}.
NON vanno consegnati i file relativi al meccanismo di autovalutazione 
(\textit{Test\_*.java, AbstractTest.java, *.sh}).
Per la consegna, eseguite l'upload dei SINGOLI file sorgente (NON un file archivio!) dalla pagina web: http://upload.di.unimi.it
nella sessione del vostro docente.

\hrule
\medskip
\begin{center}*** ATTENZIONE!!! ***\end{center}

NON VERRANNO VALUTATI GLI ELABORATI CON ERRORI DI COMPILAZIONE O 
LE CONSEGNE CHE NON RISPETTANO LE SPECIFICHE (ad esempio consegnare un 
archivio zippato è sbagliato, come anche consegnare ad un docente diverso dal 
proprio assegnato).

UN SINGOLO ERRORE DI COMPILAZIONE O DI PROCEDURA INVALIDA 
\textbf{TUTTO} 
L'ELABORATO.

\medskip


\hrule
\medskip
{\bf Per ritirarsi}
fare l'upload di un file vuoto di nome  \texttt{ritirato.txt}. Se avete caricato dei file nella sessione del docente sbagliato, caricate l\'i un file vuoto di nome  \texttt{errataConsegna.txt} e caricate poi i file nella sessione giusta.  

\hrule


%%%%%%%%%%%%%%
\end{document}
