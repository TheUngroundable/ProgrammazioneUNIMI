\documentclass[a4paper,12pt]{article}

\usepackage{verbatim}
%\usepackage[italian]{babel}
\usepackage[a4paper,left=2cm,right=2cm,top=2cm]{geometry}
%\usepackage[a4paper,left=1.9cm,right=1.9cm,top=1.5cm]{geometry}
%\usepackage[compact]{titlesec}
% \titlespacing{\section}{0pt}{*1}{*1}
% \titlespacing{\subsection}{10pt}{*1}{*1}
% \titlespacing{\subsubsection}{10pt}{*1}{*1}
%\usepackage{mdwlist}
%\setlength{\columnsep}{1.38cm}
\setlength\parindent{0pt}
%\setlength\bibsep{0pt}
\setlength{\parskip}{0pt}
\setlength{\parsep}{0pt}
%\usepackage[italian]{babel}
%\usepackage[latin1]{inputenc}
\usepackage[utf8]{inputenc}
%\usepackage{paralist}
\usepackage{graphicx}
\usepackage{paralist}

% \topmargin=-2.7cm
% \oddsidemargin=-1cm
% \evensidemargin=-1cm
% \textwidth=18cm
%\textheight=26.7cm

%%%%%%%%%%%%%%%%%%%%%%%%%%%%%%%%%%%%%%%%%%%%%%%%%%%%%%%%%%%%%%%%%%%%%%%
\title{Laboratorio di Programmazione}
% \\
% Edizione 1 - Turni A, B, C}

\date{\textit{ESAME del 25 Novembre 2016}}
%\textbf{BOZZA}}
%\makeindex

%%%%%%%%%%%%%%%%%%
\begin{document}
%\pagestyle{empty}
%\twocolumn[\begin{center}\textbf{LabProg 2012-2013 - TESTO ESAME\\
%Andrea Trentini - D.I. - UniMi\\
%05 febbraio 2013} \end{center} \vspace{.5cm}]
\maketitle

\tableofcontents
%\hrule

\hrulefill

\section*{Avvertenze}

\begin{compactitem}
\item 
Nello svolgimento dell'elaborato è possibile
usare qualunque classe delle librerie standard di Java.

\item Non è invece
ammesso l'uso delle classi del package {\tt prog} allegato al libro di
testo del Prof.~Pighizzini e impiegato nella prima parte del corso.

\item Si consiglia CALDAMENTE l'utilizzo dello script ``checker.sh'' per 
compilare ed effettuare una prima valutazione (NON esaustiva!) del proprio elaborato.
Si consiglia anche di leggere il sorgente dei \texttt{Test\_*.java} per 
capire cosa devono offrire le classi da sviluppare.

\item Ricordarsi, quando si programma: \emph{Repetita NON iuvant} o DRY (\emph{Don't Repeat Yourself}).

\item Per la procedura di \textbf{consegna} si veda in fondo al documento.

\end{compactitem}

%%%%%%%%%%%%%%%%%%
\newpage
\hrulefill
\section{ESERCIZIO FILTRO}


\textbf{===$>>>$ INIZIARE PRIMA CON QUESTO,
se non si è in grado di portare a termine questo esercizio...
NON PROSEGUIRE!!!
(la correzione del resto dell'elaborato è subordinata alla correttezza di questo primo esercizio)}
\medskip

La successione di Fibonacci $f_1, f_2, f_3 , . . . , f_n , . . .$ è una
successione di numeri naturali così definita.
\begin{compactitem}
\item $f_1 = 1$.
\item $f_2 = 1$.
\item Per ogni $n>=3, f_n = f_{n-1} + f_{n-2}$

\end{compactitem}

Ecco i primi termini della successione:
$1, 1, 2, 3, 5, 8, 13, 21, . . .$

Scrivete un programma che visualizzi una rappresentazione ``grafica''
della magnitudine dei primi $n$ termini della successione di Fibonacci.
Il programma è costituito da una sola classe di nome \textit{Fibonacci},
contenente il metodo \textit{main}, assieme eventualmente
a dei metodi ausiliari statici (a discrezione).
Il programma legge il numero intero $n$ dalla lista degli argomenti del programma.
Se la condizione $n>=1$ non è soddisfatta, il programma termina.
Poi, il programma visualizza $n$ righe, ciascuna costituita esclusivamente da asterischi consecutivi. La riga i-esima dovrà contenere esattamente $f_i$ asterischi,
per ogni $i = 1, . . . , n$.
Ciò fatto, il programma termina.
Ad esempio il comando \texttt{java Fibonacci 4} deve generare il seguente output:
\begin{verbatim}
*
*
**
***
\end{verbatim} 
E null'altro!


\newpage



%%%%%%%%%%%%%%%%%%
\hrulefill
\section{Tema d'esame}

Lo scopo dell'esercizio è realizzare un modello semplificato per la gestione di 
strade a pedaggio e non. Quindi avremo una gerarchia di strade (Strada, Autostrada)
e una gerarchia di veicoli (Veicolo, Auto, Camion) e le classi dovranno esporre metodi per il calcolo del 
pedaggio, della velocità media, etc.

Le \textbf{classi} da realizzare sono le seguenti (dettagli nelle sezioni successive):
\medskip

\begin{compactenum}
\item \texttt{Strada}: strada generica, non prevede pedaggio
%\item \texttt{StradaPedaggio}: generica strada a pedaggio

\item \texttt{Autostrada}: sottoclasse di Strada, introduce il concetto 
di ``veicolo ammesso al transito'' (deve avere una certa potenza) e di pedaggio

\item \texttt{Biglietto}: biglietto di ingresso, tiene traccia dell'orario di 
ingresso

\item \texttt{Veicolo}: veicolo generico, classe astratta

\item \texttt{Auto}: sottoclasse di Veicolo

\item \texttt{Camion}: sottoclasse di Veicolo

% \item \texttt{Moto}: sottoclasse di Veicolo

% \item \texttt{MultaException}: eccezione che rappresenta l'evento di 
% superamento del limite di velocità

\item \texttt{Main}: programma principale

\end{compactenum}

%%%%%%%%%%%%%%%%%%
\subsection{Specifica delle classi}

Le classi (\textbf{pubbliche}!) dovranno esporre almeno i metodi e costruttori \textbf{pubblici} specificati, più eventuali altri metodi e costruttori %\textit{privati}, 
se ritenuti opportuni.
%in alcuni casi le definizioni dei metodi sono incomplete
%(vanno aggiunti i tipi mancanti).
Gli attributi (campi) delle classi devono essere \textbf{privati}.
per leggere e modificarne i valori, 
creare opportunamente, e solo dove necessario, i metodi di accesso ({\tt set} e 
{\tt get}).
Se si usano classi che utilizzano tipi generici, si suggerisce  di utilizzarne 
le versioni opportunamente istanziate (es. \texttt{ArrayList<String>} invece di
\texttt{ArrayList}).
Ogni classe deve avere il metodo {\tt 
toString} che rappresenti lo stato delle istanze. 

% Alcuni controlli di coerenza vengono suggeriti nel testo, potrebbero
% essercene altri a discrezione. 
% Si consiglia di posporre
% l'implementazione dei controlli di coerenza, come ultima operazione,
% dopo aver realizzato un sistema funzionante.

%%%%%%%%%%%%%%%%%%
\subsubsection{public class Strada}

Deve definire gli attributi: \texttt{int limite (Km/h); float lunghezza 
(Km)}.

Deve definire almeno un costruttore che permetta di impostare gli attributi.

E i seguenti metodi (oltre ai get per gli attributi, NON implementare i set, 
basta il costruttore):

\begin{compactitem}

\item\texttt{public Strada(float lunghezza, int limite)} \textit{per avere l'ordine dei parametri}


\item\texttt{public float orePercorrenzaVelocitaCodice()} restituisce quante 
ore ci vogliono a percorrere tutta la lunghezza della strada andando alla 
velocità massima possibile (limite di velocità). Nota: il float restituito rappresenta
ore e frazioni di ore, ad esempio 1.20 rappresenta un'ora e 12 minuti.

\item\texttt{public float velocitaMediaDatoTempoPercorrenzaInSec(float 
percorrenza)} restituisce la velocità media in Km/h tenuta, dato un tempo di percorrenza 
effettivo espresso in secondi.

\item\texttt{public boolean superatoLimiteDatoTempoPercorrenzaInSec(float 
percorrenza)} restituisce \textit{true} se, dato il tempo di percorrenza (in secondi), si può 
supporre che il limite di velocità sia stato superato.

\end{compactitem}





%%%%%%%%%%%%%%%%%%
\subsubsection{public class Autostrada}

Sottoclasse di Strada, aggiunge la gestione del pedaggio,
la gestione dell'accesso (accedono 
solo i veicoli sopra una certa potenza) e la gestione degli ingressi (tiene 
traccia di chi è dentro, calcola se sono stati superati i limiti, etc.).

Deve definire gli attributi: \texttt{float tariffaBase (euro/Km)},
 \texttt{float 
potenzaMinimaPerAccedere (kW)} e un container (a scelta) per tenere traccia dei veicoli 
entrati.

Deve definire almeno un costruttore che permetta di impostare gli attributi.

E i seguenti metodi (oltre ai get per gli attributi, NON implementare i set, 
basta il costruttore):

\begin{compactitem}

\item\texttt{public Autostrada(int lunghezza, int limite,\\
float tariffaBase, float potenzaMinimaPerAccedere)} \textit{per avere l'ordine dei parametri}


\item \texttt{public float pedaggio(Veicolo v)} accetta solo veicoli con potenza superiore a potenzaMinima (se il veicolo non ha sufficiente potenza lancia eccezione) e 
calcola il pedaggio (in euro) in funzione di:\\
- tariffa base\\
- lunghezza della strada\\
- numero di assi del veicolo: fino a 3 assi si usa la tariffa base, oltre i 3 assi si usa 1.5*(tariffa base)
%, le moto pagano la metà

\item \texttt{public Biglietto ingresso(Veicolo v)} se il veicolo è ammesso
(controllo su potenza), lo lascia entrare ed emette un biglietto (vedi classe relativa) per il 
veicolo stesso; lancia un'eccezione se il veicolo non è ammesso.
Nota: java.lang.System.currentTimeMillis() permette di avere l'ora esatta espressa 
in millisecondi.

\item \texttt{public float uscita(Veicolo v)} se il veicolo non è presente tra 
i veicoli entrati, lancia un'eccezione; se il tempo di percorrenza (calcolato in base
all'ora attuale e all'ora di ingresso) è inferiore a 
quello ``legale'' (rispettando il limite di velocità), lancia un'eccezione,
altrimenti fa uscire il veicolo (elimina dal container), calcola il pedaggio e lo restituisce.


\item \texttt{public int quantiVeicoli()} restituisce il numero di veicoli 
attualmente in viaggio.

\item \texttt{public float potenzaMedia()} calcola e restituisce la potenza 
media dei veicoli attualmente in viaggio.

%\item \texttt{public float quanteMoto()} calcola e restituisce il numero di moto presenti.


\end{compactitem} 



%%%%%%%%%%%%%%%%%%
\subsubsection{public abstract class Veicolo}

Deve definire gli attributi:     \texttt{String targa;  int assi (numero di); 
float potenza (in kW)}. Inoltre deve avere un reference a Biglietto (vedi classe 
relativa) con i metodi set e get relativi.

\texttt{public Veicolo(String targa, int assi, float potenza)} \textit{per ordine parametri}



%%%%%%%%%%%%%%%%%%
\subsubsection{public class Auto}

Sottoclasse di Veicolo, non ha attributi aggiuntivi, implementa un costruttore 
che fissa il numero di assi a 2.

\texttt{public Auto(String targa, float potenza)} \textit{per ordine parametri}


% %%%%%%%%%%%%%%%%%%
% \subsection*{public class Moto}
% 
% Sottoclasse di Veicolo, non ha attributi aggiuntivi, implementa un costruttore 
% che fissa il numero di assi a 2 e la lunghezza a 3 metri.

%%%%%%%%%%%%%%%%%%
\subsubsection{public class Camion}


Sottoclasse di Veicolo, non ha attributi aggiuntivi, implementa un costruttore 
che fissa il numero di assi a 5 e un costruttore con numero di assi parametrico.

\texttt{public Camion(String targa, float potenza)\\
public Camion(String targa, float potenza, int assi)} \textit{per ordine parametri}


%%%%%%%%%%%%%%%%%%
\subsubsection{public class Biglietto}

Deve definire un attributo  \texttt{long timestamp} (per registrare l'orario di ingresso) e 
avere un reference a Veicolo, coi relativi set e get.

\textbf{Suggerimento:} per il timestamp usare \texttt{java.lang.System.currentTimeMillis()},
che restituisce l'ora corrente espressa in millisecondi.

%%%%%%%%%%%%%%%%%%
\subsubsection{public class Main}


Creare un main che realizzi i seguenti:

\begin{compactenum}
\item istanzi una Autostrada (attributi: 50 Km lunghezza, 100 Km/h 
limite vel., 0.20 euro/Km pedaggio, 50 kW potenza minima per accedere)
\item istanzi una serie di veicoli prendendo i dati da un file di testo (vedi 
sotto)
\item li immetta uno a uno nell'autostrada
\item stampi la situazione dell'autostrada dopo ogni ingresso
\item calcoli la potenza media dei veicoli che sono effettivamente entrati
\item li faccia uscire uno a uno dall'autostrada, stampando la situazione
dopo ogni uscita
\end{compactenum}

\textbf{Formato del file di testo:}  TipoVeicolo;Targa;Potenza

Utilizzare il seguente contenuto:

\begin{quote}
Camion;MI2121212;600\\
Auto;PV77777;60\\
Camion;CO98989;900\\
Auto;MI656565;200\\
Auto;MI43432;40\\
Camion;MI2323232;600\\
%Moto;MI373738;70
\end{quote}

(potete usare il file \texttt{autostrada.txt} allegato al tema)

%%%%%%%%%%%%%%%%%%%%%%%%%%%%%%%%%%%%%%%%%%%%%%%%%%%%%%%
\hrulefill
\section{Raccomandazioni}

Affinché l'elaborato sia valutato, è richiesto che sia le classi sviluppate che i test risultino
\textit{compilabili}. A tal fine i metodi/costruttori che non saranno sviluppati dovranno comuque avere
una implementazione fittizia come la seguente:
\begin{verbatim}
public void nomeDelMetodo () {
   throw new UnsupportedOperationException();
}
\end{verbatim}
Si suggerisce quindi di dotare da subito le classi di tutti i metodi richiesti, 
implementandoli in modo fittizio, e poi di sostituire man mano le 
implementazioni fittizie con implementazioni che rispettino le specifiche.

\hrulefill

%%%%%%%%%%%%%%%%%%
\section{Consegna}
%\marginpar{QUESTA SEZIONE E' PRESENTE PER COMPLETEZZA}
Si ricorda che le classi devono essere tutte \textit{public} e che vanno 
consegnati tutti (e soli) i file \textit{.java} prodotti.
NON vanno consegnati i \textit{.class}.
NON vanno consegnati i file relativi al meccanismo di autovalutazione 
(\textit{Test\_*.java, AbstractTest.java, *.sh}).
Per la consegna, eseguite l'upload dei SINGOLI file sorgente (NON un file archivio!) dalla pagina web: http://upload.di.unimi.it
nella sessione del vostro docente.

\hrulefill

\begin{center}*** ATTENZIONE!!! ***\end{center}

NON VERRANNO VALUTATI GLI ELABORATI CON ERRORI DI COMPILAZIONE O 
LE CONSEGNE CHE NON RISPETTANO LE SPECIFICHE (ad esempio consegnare un 
archivio zippato è sbagliato, come anche consegnare ad un docente diverso dal 
proprio assegnato).

UN SINGOLO ERRORE DI COMPILAZIONE O DI PROCEDURA INVALIDA 
\textbf{TUTTO} 
L'ELABORATO.
\medskip

\hrulefill

{\bf Per ritirarsi}
fare l'upload di un file vuoto di nome  \texttt{ritirato.txt}. Se avete caricato 
dei file nella sessione del docente sbagliato, caricate lì un file vuoto di 
nome  \texttt{errataConsegna.txt} e caricate poi i file nella sessione giusta. 

\hrulefill

%%%%%%%%%%%%%%
\end{document}
