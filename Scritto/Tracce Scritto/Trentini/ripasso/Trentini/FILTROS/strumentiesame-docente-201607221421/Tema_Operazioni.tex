\documentclass[12pt]{article}

\usepackage{verbatim}
%\usepackage[italian]{babel}
\usepackage[a4paper,left=2cm,right=2cm,top=2cm]{geometry}
%\usepackage[a4paper,left=1.9cm,right=1.9cm,top=1.5cm]{geometry}
%\usepackage[compact]{titlesec}
% \titlespacing{\section}{0pt}{*1}{*1}
% \titlespacing{\subsection}{10pt}{*1}{*1}
% \titlespacing{\subsubsection}{10pt}{*1}{*1}
%\usepackage{mdwlist}
%\setlength{\columnsep}{1.38cm}
\setlength\parindent{0pt}
%\setlength\bibsep{0pt}
\setlength{\parskip}{0pt}
\setlength{\parsep}{0pt}
%\usepackage[italian]{babel}
%\usepackage[latin1]{inputenc}
\usepackage[utf8]{inputenc}
\usepackage{graphicx}
\usepackage{paralist}

% \topmargin=-2.7cm
% \oddsidemargin=-1cm
% \evensidemargin=-1cm
% \textwidth=18cm
%\textheight=26.7cm

%%%%%%%%%%%%%%%%%%%%%%%%%%%%%%%%%%%%%%%%%%%%%%%%%%%%%%%%%%%%%%%%%%%%%%%
\title{Laboratorio di Programmazione\\
\textit{ESAME del 11 luglio 2016}}

\date{}

%\makeindex

%%%%%%%%%%%%%%%%%%
\begin{document}
%\pagestyle{empty}
%\twocolumn[\begin{center}\textbf{LabProg 2012-2013 - TESTO ESAME\\
%Andrea Trentini - D.I. - UniMi\\
%05 febbraio 2013} \end{center} \vspace{.5cm}]
\maketitle

%  \tableofcontents
%  \hrule

\hrulefill

\textbf{Avvertenze}

\begin{compactitem}
\item 
Nello svolgimento dell'elaborato è possibile
usare qualunque classe delle librerie standard di Java.

\item Non è invece
ammesso l'uso delle classi del package {\tt prog} allegato al libro di
testo del Prof.~Pighizzini e impiegato nella prima parte del corso.

\item Si consiglia CALDAMENTE l'utilizzo dello script ``checker.sh'' (se non è eseguibile renderlo tale col comando \texttt{chmod}) per 
compilare ed effettuare una prima valutazione del proprio elaborato.
Si consiglia anche di leggere il sorgente dei \texttt{Test\_*.java} per 
capire cosa devono offrire le classi da sviluppare.

\item Ricordarsi, quando si programma: \emph{Repetita NON iuvant} o DRY (\emph{Don't Repeat Yourself}).

\item Un corpo di metodo (escluso un \texttt{main}) più lungo di una decina di righe è un buon indizio di ``strada sbagliata''

\item Se avete dubbi sulla interpretazione del testo chiedete!

\end{compactitem}


\hrulefill
%\medskip

%%%%%%%%%%%%%%%%%%
\section*{ESERCIZIO FILTRO}


\textbf{===$>>>$ INIZIARE PRIMA CON QUESTO,
se non si è in grado di portare a termine questo esercizio...
NON PROSEGUIRE!!!
(la correzione del resto dell'elaborato è subordinata alla correttezza di questo primo esercizio)}

\medskip

% Realizzare una classe \texttt{Stat}, dotata del solo \texttt{main},
% che letta da linea di comando la distribuzione di una variabile
% casuale discreta con valori interi compresi fra 18 e 30, calcoli e
% visualizzi valore atteso e varianza.

Realizzare una classe \texttt{Stat}, dotata del solo \texttt{main},
che letta da linea di comando una serie di coppie di numeri $v_i$ $p_i$
che rappresentano ciascuna un voto e la sua probabilità, calcoli e visualizzi il valore atteso, secondo la formula fornita sotto.
\begin{compactitem}
\item i valori $v_i$ sono di tipo \texttt{int} compresi tra 18 e 30.
\item i valori $p_i$ sono di tipo \texttt{float} compresi tra 0 e 1 e tali che $\sum_{i} p_i = 1.0$
\end{compactitem}

% La distribuzione è descritta da coppie contigue di numeri $x_i$ $p_i$,
% di cui il primo di tipo \texttt{int} ed il secondo di tipo
% \texttt{float}, tali che
% $\sum_{i} p_i = 1.0$

Ecco due esempi:
\begin{verbatim}
18 0.15 21 0.15 22 0.05 24 0.05 25 0.25 26 0.05 28 0.18 30 0.12
24 0.5 25 0.2 30 0.3
\end{verbatim} 
%18*0.15+21*0.15+22*0.05+24*0.05+25*0.25+26*0.05+28*0.18+30*0.12
Si ricorda che il valore atteso ($E$) \`e definito da
$$E = \sum_{i} v_i \cdot p_i$$
% mentre la varianza \`e definita da
% $$V = \sum_{i} (v_i - E)^2 \cdot p_i$$

Qualora uno dei valori $v_i$ inseriti non sia nel range [18,30] o uno dei $p_i$ non sia nel range [0,1]
l'applicazione terminerà visualizzando un messaggio d'errore.
A parte ciò, SI ASSUMA che l'input (gli argomenti dell'applicazione)
sia CORRETTO sintatticamente e semanticamente, e non vuoto.
Ecco un possibile \textbf{esempio} di esecuzione:

\begin{verbatim}
$ java Stat  18 0.15 21 0.15 22 0.05 24 0.05 25 0.25 26 0.05 28 0.18 30 0.12
E : 24.340002
\end{verbatim}
%V : 14.4844



\hrulefill
%\medskip

\section*{Tema d'esame}

Lo scopo dell'esercizio è realizzare un insieme di operandi e operatori. Gli \textit{operandi} sono gli argomenti, i valori da trattare, mentre le \textit{operazioni} calcolano delle funzioni semplici su insiemi di \textit{operandi}. 

Le \textbf{classi} da realizzare sono le seguenti:
\medskip

\begin{compactenum}
\item \texttt{Operando}: rappresenta (``wrapper'') un valore double

\item \texttt{Operazione}: classe ASTRATTA che rappresenta una generica operazione da applicare ad un insieme di operandi

\item \texttt{Somma}: sottoclasse di \texttt{Operazione}, calcola la somma degli operandi che contiene

\item \texttt{Massimo}: sottoclasse di \texttt{Operazione}, calcola il massimo tra gli operandi che contiene

\item \texttt{Media}: sottoclasse di \texttt{Operazione}, calcola la media degli operandi che contiene

\item \texttt{Moltiplicazione}: sottoclasse di \texttt{Operazione}, calcola il prodotto degli operandi che contiene

\item \texttt{SommaUnici}: sottoclasse di \texttt{Operazione}, calcola la somma degli operandi che contiene, ESCLUDENDO le ripetizioni

\end{compactenum}

%%%%%%%%%%%%%%%%%%
\section*{Specifica delle classi}

Le classi (\textbf{pubbliche}!) dovranno esporre almeno i metodi e costruttori \textbf{pubblici} indicati, più eventuali altri metodi e costruttori %\textit{privati}, 
se ritenuti opportuni.
%in alcuni casi le definizioni dei metodi sono incomplete
%(vanno aggiunti i tipi mancanti).
Gli attributi (campi) delle classi devono essere dichiarati \textbf{privati}.
Per leggere e modificarne i valori, 
creare opportunamente, e solo dove indicato, i metodi di accesso ({\tt set} e 
{\tt get}).
Se si usano classi che utilizzano tipi generici, si suggerisce  di utilizzarne 
le versioni opportunamente istanziate (es. \texttt{ArrayList<String>} invece di
\texttt{ArrayList}).
Ogni classe deve avere il metodo {\tt 
toString} che descriva lo stato delle istanze. 

% Alcuni controlli di coerenza vengono suggeriti nel testo, potrebbero
% essercene altri a discrezione. 
% Si consiglia di posporre
% l'implementazione dei controlli di coerenza, come ultima operazione,
% dopo aver realizzato un sistema funzionante.

%%%%%%%%%%%%%%%%%%%%%%%%%%%%%%%%%%%%%%%
\section{Operando (Comparable)}

Rappresenta un generico valore \texttt{double}. Implementa l'interfaccia \texttt{Comparable<Operando>}
Deve implementare almeno i seguenti metodi:

\begin{compactitem}
\item \texttt{public Operando(double valore)} costruttore  che accetta il valore interno iniziale
\item \texttt{public Operando()} costruttore  che imposta a 0 il valore interno
\item \texttt{public double getValore()} restituisce il valore corrente
\item \texttt{public String toString()} ritorna una rappresentazione dello stato dell'oggetto
\end{compactitem}


%%%%%%%%%%%%%%%%%%%%%%%%%%%%%%%%%%%%%%%
\section{Operazione}

Classe \textbf{astratta}, ``contiene'' un insieme (potenzialmente infinito) di operandi, le sue sottoclassi completeranno l'implementazione realizzando il calcolo effettivo.
Deve implementare almeno i seguenti metodi:

\begin{compactitem}
 \item \texttt{public void addOperando(Operando o)}: aggiunge un operando all'insieme corrente, NON effettua il calcolo dell'operazione

 \item \texttt{public int getNumeroOperandi()}: restituisce il numero degli operandi attualmente contenuti
 
 \item \texttt{public List<Operando> getOperandi()}: restituisce una struttura dati compatibile con 
\texttt{List<Operando>} contenente tutti gli operandi attualmente presenti in operazione

%\item \texttt{public double getRisultato()}: restituisce il risultato corrente (cioè riferito all'ultima invocazione di \texttt{calcola()}) dell'operazione, NON effettua il calcolo dell'operazione

%\item \texttt{protected void setRisultato(double r)}: imposta il risultato dell'operazione, NON effettua il calcolo dell'operazione

\item \texttt{public void sort()}: ordina gli operandi in base al loro valore

\item \texttt{public String toString()}: restituisce una forma testuale dello stato dell'oggetto

\item \texttt{public double ultimo()}: restituisce il valore dell'ultimo operando del contenitore

\item \texttt{public abstract double calcola()}: efffettua il calcolo (e restituisce il valore relativo) effettivo, verrà implementato nelle sottoclassi, deve lanciare un'eccezione se l'operazione non ha operandi su cui lavorare
\end{compactitem}

%%%%%%%%%%%%%%%%%%%%%%%%%%%%%%%%%%%%%%%
\section{Somma}

Sottoclasse di Operazione, calcola la somma degli operandi, ad es. se la lista dei valori è: 3, 5, 8, 3, 3 il risultato diventa 3+5+8+3+3=22.


%%%%%%%%%%%%%%%%%%%%%%%%%%%%%%%%%%%%%%%
\section{Massimo}

Sottoclasse di Operazione, calcola il massimo tra gli operandi, ad es. se la lista dei valori è: 3, 5, 8 il risultato diventa 8.

%%%%%%%%%%%%%%%%%%%%%%%%%%%%%%%%%%%%%%%
\section{Media}

Sottoclasse di Operazione, calcola la media degli operandi, ad es. se la lista dei valori è: 4, 9, 8, 7, 3 il risultato diventa (4+9+8+7+3)/5=6,2.



%%%%%%%%%%%%%%%%%%%%%%%%%%%%%%%%%%%%%%%
\section{Moltiplicazione}

Sottoclasse di Operazione, calcola il prodotto degli operandi, ad es. se la lista dei valori è: 3, 5, 8 il risultato diventa 3*5*8=120.

%%%%%%%%%%%%%%%%%%%%%%%%%%%%%%%%%%%%%%%
\section{SommaUnici}

Sottoclasse di Operazione, calcola la somma degli operandi considerando \textbf{una sola volta} gli operandi uguali, ad es. se la lista dei valori è: 3, 5, 8, 3, 3 il risultato diventa 3+5+8=16.


%%%%%%%%%%%%%%%%%%%%%%%%%%%%%%%%%%%%%%%%%%%%%%%%%%%%%%%
% \hrulefill
% \section*{Raccomandazioni per la consegna}
% 
% Affinché l'elaborato sia valutato, è richiesto che sia le classi sviluppate che i test risultino
% \textit{compilabili} SENZA ERRORI. A tal fine i metodi/costruttori che non saranno sviluppati dovranno comuque avere
% una implementazione fittizia come la seguente:
% \begin{verbatim}
% public <tipo> <nomeDelMetodo> (<lista parametri>) {
%    throw new UnsupportedOperationException();
% }
% \end{verbatim}
% Si suggerisce quindi di dotare da subito le classi di tutti i metodi richiesti, 
% implementandoli in modo fittizio, e poi di sostituire man mano le 
% implementazioni fittizie con implementazioni che rispettino le specifiche.

\hrulefill

%%%%%%%%%%%%%%%%%%
\section*{Consegna}
%\marginpar{QUESTA SEZIONE E' PRESENTE PER COMPLETEZZA}
Si ricorda che:
\begin{compactitem}
 \item le classi devono essere tutte \textit{public}
 \item vanno consegnati tutti (e soli) i file \textit{.java} prodotti
 \item si consiglia di fare upload successivi e frequenti, i docenti vedono solo l'ultima versione di ogni file
 \item NON va consegnato un file ``archivio''! (NO tar, zip, etc.)
 \item NON vanno consegnati i \textit{.class}
 \item NON vanno consegnati i file relativi al meccanismo di autovalutazione (\textit{Test\_*.java, AbstractTest.java, *.sh})
 \item eseguite l'upload dei SINGOLI file sorgente (http://upload.di.unimi.it) nella sessione ``Trentini''
\end{compactitem}

\hrulefill

\begin{center}*** ATTENZIONE!!! ***\end{center}

NON VERRANNO VALUTATI GLI ELABORATI CON ERRORI DI COMPILAZIONE O 
LE CONSEGNE CHE NON RISPETTANO LE SPECIFICHE (ad esempio consegnare un 
archivio zippato è sbagliato).%, come anche consegnare ad un docente diverso dal proprio assegnato).

UN SINGOLO ERRORE DI COMPILAZIONE O DI PROCEDURA INVALIDA 
\textbf{TUTTO} 
L'ELABORATO.
\medskip

\hrulefill

{\bf Per ritirarsi}
fare l'upload di un file vuoto di nome  \texttt{ritirato.txt}.
% Se avete caricato 
% dei file nella sessione del docente sbagliato, caricate lì un file vuoto di 
% nome  \texttt{errataConsegna.txt} e caricate poi i file nella sessione giusta. 

\hrulefill

%%%%%%%%%%%%%%
\end{document}

